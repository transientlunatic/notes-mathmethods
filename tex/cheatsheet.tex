\documentclass[9pt,landscape, a4paper]{article}
\usepackage{multicol}
\usepackage{calc}
\usepackage{ifthen}
\usepackage[landscape]{geometry}
\usepackage{hyperref}
\usepackage{danielphysics}
\usepackage{xcolor,colortbl}
\usepackage{booktabs}
\mathcode`~="8000
\def~#1{\ensuremath{\sb{\mathrm{#1}}}}%
\def\parsec{\ensuremath{\mathrm{p}}}

\makeatletter
\def\multicols@string{multicols}
\makeatother


% To make this come out properly in landscape mode, do one of the following
% 1.
%  pdflatex latexsheet.tex
%
% 2.
%  latex latexsheet.tex
%  dvips -P pdf  -t landscape latexsheet.dvi
%  ps2pdf latexsheet.ps


% If you're reading this, be prepared for confusion.  Making this was
% a learning experience for me, and it shows.  Much of the placement
% was hacked in; if you make it better, let me know...


% 2008-04
% Changed page margin code to use the geometry package. Also added code for
% conditional page margins, depending on paper size. Thanks to Uwe Ziegenhagen
% for the suggestions.

% 2006-08
% Made changes based on suggestions from Gene Cooperman. <gene at ccs.neu.edu>


% To Do:
% \listoffigures \listoftables
% \setcounter{secnumdepth}{0}


% This sets page margins to .5 inch if using letter paper, and to 1cm
% if using A4 paper. (This probably isn't strictly necessary.)
% If using another size paper, use default 1cm margins.
\ifthenelse{\lengthtest { \paperwidth = 11in}}
	{ \geometry{top=.5in,left=.5in,right=.5in,bottom=.5in} }
	{\ifthenelse{ \lengthtest{ \paperwidth = 297mm}}
		{\geometry{top=1cm,left=1cm,right=1cm,bottom=1cm} }
		{\geometry{top=1cm,left=1cm,right=1cm,bottom=1cm} }
	}

% Turn off header and footer
\pagestyle{empty}
 

% Redefine section commands to use less space
\makeatletter
\renewcommand{\section}{\@startsection{section}{1}{0mm}%
                                {-1ex plus -.5ex minus -.2ex}%
                                {0.5ex plus .2ex}%x
                                {\normalfont\large\bfseries}}
\renewcommand{\subsection}{\@startsection{subsection}{2}{0mm}%
                                {-1explus -.5ex minus -.2ex}%
                                {0.5ex plus .2ex}%
                                {\normalfont\normalsize\bfseries}}
\renewcommand{\subsubsection}{\@startsection{subsubsection}{3}{0mm}%
                                {-1ex plus -.5ex minus -.2ex}%
                                {1ex plus .2ex}%
                                {\normalfont\small\bfseries}}
\makeatother

% Define BibTeX command
\def\BibTeX{{\rm B\kern-.05em{\sc i\kern-.025em b}\kern-.08em
    T\kern-.1667em\lower.7ex\hbox{E}\kern-.125emX}}

% Don't print section numbers
\setcounter{secnumdepth}{0}


\setlength{\parindent}{0pt}
\setlength{\parskip}{0pt plus 0.5ex}


% -----------------------------------------------------------------------

\begin{document}

\raggedright
\footnotesize
\begin{multicols*}{4}


% multicol parameters
% These lengths are set only within the two main columns
%\setlength{\columnseprule}{0.25pt}
\setlength{\premulticols}{1pt}
\setlength{\postmulticols}{1pt}
\setlength{\multicolsep}{1pt}
\setlength{\columnsep}{2pt}

\begin{center}
     \Large{\textbf{Mathematical Methods II}} \\ \small{Daniel Williams} \vspace{0.3cm}\hline
\end{center}

\section{Summation Convention}
\label{sec:summation-convention}

\begin{equation} 
  \vec{A} = A_i \vec{e_i} = \sum_{i=1}^3 A_i\vec{e_i} 
  \tag{Summation}
\end{equation}

\begin{equation}
  \label{eq:dotprod}
  \vec{a} \cdot \vec{b} = a_\theta b_\theta
  \tag{Dot}
\end{equation}

\begin{equation}
  \label{eq:totaldifferential}
  \dd{f} = \frac{\partial f}{\partial x_\theta} \dd{x_\theta}
  \tag{Total Diff}
\end{equation}

\begin{equation}
  \label{eq:matmult}
  a_{ij} = b_{i \theta} c_{\theta j}
  \tag{Matrix Mult.}
\end{equation}

\begin{equation}
  \label{eq:crossprod1}
  [ \vec a \times \vec b]_i = \epsilon_{i \theta \phi} a_{\theta} b_{\phi}
  \tag{Cross}
\end{equation}

\begin{equation}
  \label{eq:deltalevi}
  \epsilon_{\theta j k } \epsilon_{\theta l m} = \delta_{jl} \delta_{k m} - \delta_{j m} \delta_{l k}
  \tag{\epsilon \to \delta}
\end{equation}

\begin{equation}
  \label{eq:grad}
  \nabla = \vec e_i \frac{\partial }{\partial x_i}
  \tag{Grad}
\end{equation}

\begin{equation}
  \label{eq:divergence}
  \nabla \cdot \vec{A}(\vec{r}) = \frac{\partial A_i(\vec{r})}{\partial x_i}
  \tag{Div}
\end{equation}

\begin{equation}
  \label{eq:curl}
  \nabla \times \vec{A}(\vec{r}) = \epsilon_{ijk} \frac{\partial A_j(\vec{r})}{\partial x_i} \vec{e_k}
  \tag{Curl}
\end{equation}

\section{Curvilinear Coordinates}
\label{sec:curv-coord}
\begin{center}
\begin{tabular}{l | l}
  Cylindrical        & Spherical                        \\\hline
$x = r \cos(\theta)$ & $ x = r \sin(\theta) \cos(\phi)$ \\
$y = r \sin(\theta)$ & $ y = r \sin(\theta) \sin(\phi)$ \\
$z = z$              & $z = r \cos(\theta)$
\end{tabular}
\end{center}
\begin{equation}
  \label{eq:4}
  \pdv{\vec{r}}{q_i} = h_{q_i} \vec{e}_{q_i}
\tag{Scale factor}
\end{equation}
\begin{equation}
  \label{eq:1}
  \dd{s}^2 = \dd{r} \cdot \dd{r} = \sum g_{ij} \dd{q_i} \dd{q_j}
  \tag{Line Element}
\end{equation}
\begin{equation}
  \label{eq:3}
  \dd{V} = \sum_i h_{q_i} \dd{q_i}
\tag{Volume Element}
\end{equation}

\section{Special Functions}
\label{sec:special-functions}

(Most Summarised in table below)

Spherical Harmonics 


\end{multicols*}
\multicolinterrupt{ 
%\begin{table}
\begin{tabular}{p{.7in} p{2.7in}p{2in}p{2in}p{2.5in}}
\toprule
    & Legendre & Bessel & Hermite & Laguerre \\ 
\toprule
%%% Differential Equations
%\rowcolor{muted-egg}
ODE 
& %%% Legendre
$ \dv{x} \qty[ (1-x)^2 \dv{x} P_n(x) ] + n(n+1) P_n(x) = 0 $
& %%% Bessel
$ x^2 \dv[2]{y}{x} + x \dv{y}{x} + (x^2 - \alpha^2)y = 0 $
& %%% Hermite
$ \dv[2]{y}{x} - 2x \dv{y}{x} + 2n y = 0$
& %%% Laguerre
$ x \dv[2]{y}{x} + (1-x) \dv{y}{x} + n y = 0$ \\
\midrule
%%% Rodrigues Functions
Rodrigues
& %%% Legendre
$ P_n(x) = \frac{1}{2^n\ n!} \frac{\difp{n}{}}{\difp{n}{x}} \left[ (x^1-1)^n \right] $
& %%% Bessel

& %%% Hermite
$ H_n(x) = (-1)^n e^{x^2} \dv[n]{x} \qty(e^{-x^2}) $
& %%% Laguerre
$ L_n(x) = \frac{e^x}{n!} \dv[n]{x} \qty(x^n e^{-x} )$ \\
\midrule
%%% Generating Functions
Generating
& %%% Legendre
$ \frac{1}{\sqrt{1- 2xt +t^2}} = \sum^{\infty}_{n=0} P_n(x) t^n $
& %%% Bessel
$ \exp(\frac{x}{2t}(t^2-1)) = \sum_{\nu=-\infty}^{\infty} J_{\nu}(x) t^{\nu} $
& %%% Hermite
$ e^{-t^2 + 2tx} = \sum^\infty_{n=0} H_n(x) \frac{t^n}{n!} $
& %%% Laguerre
$ \frac{1}{1-t} \exp( - \frac{xt}{(1-t)}) = \sum_{n=0}^{\infty} L_n(x) t^n$ \\
\midrule
%%% Parity
Symmetry
& %%% Legendre
$  P_n(-x) = (-1)^n P_n(x) $
& %%% Bessel

& %%% Hermite
$ H_n(-x) = (-1)^n H_n(x) $
& %%% Laguerre
\\
\midrule
%%% Range
Range
& %%% Legendre
$ [-1, 1] $
& %%% Bessel
$ [0, \infty)$
& %%% Hermite
$ (-\infty, \infty) $
& %%% Laguerre
$ [0, \infty)$ \\ \midrule
%%% Recurrence
Recurrence
& %%% Legendre
$ (2n+1) x P_n(x) = (n+1) P_{n+1}(x) + nP_{n-1}(x) $

$P^{\prime}_{n+1}(x) + P^{\prime}_{n-1}(x) = 2x P_n^{\prime}(x) +
P_n(x)$
& %%% Bessel
$ J_{\nu-1}(x) + J_{\nu+1}(x) = \frac{2 \nu}{x} J_{\nu}(x)$ 

$J_{\nu-1}(x) - J_{\nu+1}(x) = 2J_{\nu}^{\prime}(x)$
& %%% Hermite
$ H_{n+1}(x) = 2x H_n(x) - 2n H_{n-1}(x)$

$H_n^{\prime}(x) = 2n H_{n-1}(x)$
& %%% Laguerre
$(n+1) L_{n+1}(x) = (2n +1 -x) L_n(x) - nL_{n-1}(x)$
$xL^{\prime}_n(x) = nL_n(x) - nL_{n-1}(x)$
\\
\midrule
%%% Assoc
Assoc
& %%% Legendre
$ P_n^m(x) = (1-x^2)^{\frac{m}{2}} \frac{\dif{}^m}{\dif{x}^m} P_n(x) $
& %%% Bessel

& %%% Hermite

& %%% Laguerre
$ L_n^k(x) = (-1)^n \dv[k]{x} L_{n+k}(x) $
\\
\bottomrule
\end{tabular}
%\end{table}
}


\end{document}

%%% Local Variables: 
%%% mode: latex
%%% TeX-master: t
%%% End: 
