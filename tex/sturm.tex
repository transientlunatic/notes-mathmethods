Linear operators can be further generalised to a wider class of
\emph{differential operators}, ${\cal L}$ which act on eigenfunctions
$y_i(x)$ which have eigenvalues $\lambda_i$, and weighted by a weight
function $\rho(x)$.

\begin{definition}[Sturm-Liouville Operator]
  The Sturm-Liouville operator is defined as 
  \[ {\cal L} = - \qty( p(x) \dv[2]{x} + r(x) \dv{x} + q(x) ) \]
\end{definition}

\begin{definition}[Sturm-Liouville Equation]
  The Sturm-Liouville equation is a differential equation of the form
  \[ {\cal L} y = \lambda \rho(x) y \]
\end{definition}

These can be simplified if $r(x) = p^{\prime}(x)$, when
\[ {\cal L} = - \qty( \dv{x} \qty( p(x) \dv{x} ) + q(x) ) \]
and so
\[ {\cal L} y = - (py^{\prime})^{\prime} - qy \]
 
Provided we set approporiate conditions on the functions $p(x), q(x)$,
and, $\rho(x)$, and use appropriate boundary conditions on a range
$[a,b]$, we can make a number of assertions about the SL-equation.

\begin{theorem}[Properties of the Sturm-Liouville Operator]
\begin{subtheorem}
\label{the:hermitiansloper}
  The Sturm-Liouville Operator is Hermitian over the range $[a,b]$
\end{subtheorem}
\begin{proof}
  An operator is Hermitian over the range $[a,b]$ if
  \[ \int_a^b f^{*}(x) \qty[ {\cal L} g(x)] \dd{x} = \int_a^b
  \qty[{\cal L} f(x)]^{*} g(x) \dd{x} \] In the case of the SL
  operator, \[ {\cal L} y = - \qty( p y^{\prime})^{\prime} - qy \]
  Applying ${\cal L}$ to $y_i$, and premultiplying by $y_i^{*}$, then
  integrating over $[a,b]$,
  \begin{align*}
    \int_b^a y_i^{*} {\cal L} y_i \dd{x} &= - \int_a^b \qty( y_i^{*} (p y_j^{\prime})^{\prime} + y_i^{*} q y_j ) \dd{x} \\
&= - \int_a^b y_i^{*} \qty(p y^{\prime}_j)^{\prime} \dd{x} - \int_a^b y_i^{*} q y_j \dd{x}
  \end{align*}
  The first integral can be integrated by parts to yield
  \[ \int_a^b y_i^{*} \qty(p y^{\prime}_j)^{\prime} \dd{x} = -
  \qty[y_i^{*} p y^{\prime}_j]_a^b + \int_a^b \qty(y_i^{*})^{\prime} p
  y_j^{\prime} \dd{x} \] We set the boundary conditions to set
  \[ \qty[y_i^{*} p y^{\prime}_j]_a^b = 0 \] This leaves the integral,
  which can be solved by integrating again,
  \[ \int_a^b \qty(y_i^{*})^{\prime} p y_j^{\prime} \dd{x} =
  \qty[y_i^{*} p y^{\prime}_j]_a^b = 0 - \int_a^b \qty(
  \qty(y_i^{*})^{\prime} p )^{\prime} y_j \dd{x} \] Rearranging, and
  returning to the original integral,
  \begin{align*}
    \int_b^a y_i^{*} {\cal L} y_i \dd{x} &= - \int_a^b \qty[ y_j \qty( p (y_i^{*})^{\prime} )^{\prime} + y_j q y_i^{*}] \dd{x} \\
\int_a^b y_i^{*} \qty( {\cal L} y_j ) \dd{x} &= \int_a^b \qty( {\cal L} y_i)^{*} y_j \dd{x}
  \end{align*}
\end{proof}
\begin{subtheorem}
  The eigenvalues of the Sturm-Liouville Operator are real.
\end{subtheorem}
\begin{proof}q From theorem \ref{the:hermitiansloper} the operator is
  known to be Hermitian.  From theorem \ref{the:eigenvaluehermitian}
  we know that the eigenvalues of a Hermitian matrix are real.
\end{proof}
\begin{subtheorem}
  The eigenvalues of the Sturm-Liouville Operator form an ordered set $\lambda_1 < \lambda_2 < \cdots < \lambda_n$.
\end{subtheorem}
\begin{subtheorem}
  The eigenfunctions of the Sturm-Liouville Operator, $y_i(x)$ have
  $i-1$ zeros over the range $[a,b]$.
\end{subtheorem}
\begin{subtheorem}
  The normalised eigenfunctions, $y_i(x)$ of the Sturm-Liouville operator form an orthogonal basis,
  \[ \braket{y_i}{y_j} = \delta_{ij} \]
\end{subtheorem}
\end{theorem}

%%% Local Variables: 
%%% mode: latex
%%% TeX-master: "notes.pdf"
%%% End: 
