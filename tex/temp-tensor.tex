\documentclass{mn2e}
\usepackage{danielphysics}
\begin{document}
\section{Cartesian Tensors}
\label{sec:carttensor}

Tensors generalise the concepts of scalars and vectors to higher
orders. They represent a short-hand way of describing complex physical
quantities which would otherwise require an array of vectors.
\begin{example}
  \emph{The stress tensor}     \\
  To describe all of the forces acting on a three-dimensional beam we
  need 9 components.  Consider a plane perpendicular to the
  $x$-direction. The force on this plane has three components, and
  likewise on the surfaces perpendicular to $y$ and $z$. These
  quantities can be described by a tensor of rank 2,
  \begin{equation*}
    \begin{pmatrix}
      F_{xx} & F_{xy} & F_{xz} \\
      F_{yx} & F_{yy} & F_{yz} \\
      F_{zx} & F_{zy} & F_{zz}
    \end{pmatrix}
  \end{equation*}
\end{example}
\end{document}
