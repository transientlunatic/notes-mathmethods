% \documentclass{dwnotes}
% \usepackage{danielphysics}
%\begin{document}
\section{Multilinear Algebra and Tensors}
\label{sec:multilin}

\begin{definition}[Covector]
  A covector (linear form, one-form, or linear functional) is a linear
  map from a vector space to its field of scalars.
\end{definition}
\begin{definition}[Dual Vector Space]
  The set of all covectors which map a vector space $\vs{V}$ to its
  field $F$ form a vector space over $F$, which is called the dual
  space of $\vs{V}$, and denoted $\vs{V^{*}}$.
\end{definition}
\begin{example}
  Suppose that vectors in the real vector space $\mathbb{R}^n$ are
  represented as column matrices,
\[ \vec{x} = 
\begin{pmatrix}
  x_1 \\ x_2 \\ \vdots \\ x_n 
\end{pmatrix} \]
Then any linear functional can be written in these coordinates as the sum
\[ f(\vec{x}) = a_1 \vec{x}_1 + \cdots + a_n \vec{x}_n \]
can also be represented as a row matrix,
\[ f =
\begin{pmatrix}
  a_1, & a_2, & \cdots, & a_n 
\end{pmatrix}
\]
\end{example}

\begin{definition}[Contravariant Tensor]
  Suppose $n$ quantities in a basis $\vec{x}^{1}, \dots, \vec{x}^n$ are
  related to another $n$ quantities, $A^{\prime}^1, \dots,
  A^{\prime}^n$ in a coordinate system $x^{\prime}^1, \dots,
  x^{\prime}^n$ by a set of transformations
  \[ A^{\prime}^p = \pdv{x^{\prime p}}{x^{\prime q}} A^q \] Then $A^1,
  \dots, A^n$ are the components of a first rank contravariant tensor.
\end{definition}

\begin{definition}[Covariant Tensor]
  Suppose $n$ quantities in a basis $\vec{x}_{1}, \dots, \vec{x}_n$ are
  related to another $n$ quantities, $A^{\prime}_1, \dots,
  A^{\prime}_n$ in a coordinate system $x^{\prime}_1, \dots,
  x^{\prime}_n$ by a set of transformations
  \[ A^{\prime}_p = \pdv{x^{\prime p}}{x^{\prime q}} A_q \] Then $A_1,
  \dots, A_n$ are the components of a first rank covariant tensor.
\end{definition}
An intuitive introduction to tensors is via \emph{dyads}, which were
historically the precursors of tensors. 
\begin{definition}[Dyadic Product]
  Consider two vectors, $\vec{u}, \vec{v} \in \vs{V}$, then their
  dyadic product, denoted $\dyad{uv}$, can be represented by the
  sum 
  \begin{align*}
    \label{eq:dyadicproduct}
       \dyad{ab} =           & a_1 b_1 \dyad{ii} + a_2 b_1 \dyad{ij} + a_1 b_3 \dyad{ik} \\ 
       +                     & a_2 b_1 \dyad{ji} + a_2 b_2 \dyad{jj} + a_2 b_3 \dyad{jk} \\ 
       +                     & a_3 b_1 \dyad{ki} + a_3 b_2 \dyad{kj} + a_3 b_3 \dyad{kk}
  \end{align*}
with the standard basis Dyads being of the form
\begin{equation*} 
\begin{matrix}
\dyad{ii} =\begin{pmatrix} 1 & 0 & 0                                                     \\ 0 & 0 & 0 \\ 0 & 0 & 0 \end{pmatrix} & 
\dyad{ij} =\begin{pmatrix} 0 & 1 & 0                                                     \\ 0 & 0 & 0 \\ 0 & 0 & 0 \end{pmatrix} & 
\dyad{ik} =\begin{pmatrix} 0 & 0 & 1                                                     \\ 0 & 0 & 0 \\ 0 & 0 & 0 \end{pmatrix} \\
\dyad{ji} =\begin{pmatrix} 0 & 0 & 0                                                     \\ 1 & 0 & 0 \\ 0 & 0 & 0 \end{pmatrix} & 
\dyad{jj} =\begin{pmatrix} 0 & 0 & 0                                                     \\ 0 & 1 & 0 \\ 0 & 0 & 0 \end{pmatrix} & 
\dyad{jk} =\begin{pmatrix} 0 & 0 & 0                                                     \\ 0 & 0 & 1 \\ 0 & 0 & 0 \end{pmatrix} \\
\dyad{ki} =\begin{pmatrix} 0 & 0 & 0                                                     \\ 0 & 0 & 0 \\ 1 & 0 & 0 \end{pmatrix} & 
\dyad{kj} =\begin{pmatrix} 0 & 0 & 0                                                     \\ 0 & 0 & 0 \\ 0 & 1 & 0 \end{pmatrix} & 
\dyad{kk} =\begin{pmatrix} 0 & 0 & 0                                                     \\ 0 & 0 & 0 \\ 0 & 0 & 1 \end{pmatrix}
\end{matrix}
\end{equation*}
In terms of dyads, both the outer product, $\vec{a} \vec{b}^{\rm T}$
and the tensor product, $\vec{a} \otimes \vec{b}$ are the same quantity.
\end{definition}

\begin{definition}[Tensor]
  A tensor of type $(n, m-n)$ is an assignment of a multi-dimensional array, 
  \[ T_{i_{n+1} \cdots i_m}^{i_1 \cdots i_n} \qty[\vec{f}] \] to each
  basis $\vec{f} = (\vec{e}_1, \dots \vec{e}_n)$, such that, if a
  change of basis is applied,
\[ \vec{f} \to R \vec{f} = (R^i_1 \vec{e}_1, \dots, R^i_n \vec{e}_n ) \]
then the array obeys the transform law
\[ T_{i_{n+1} \cdots i_m}^{i_1 \cdots i_n} \qty[ \vec{f} \cdot R] = (R^{-1})^{i_1}_{j_1}  R^{j_{n+1}}_{i_{n+1}}  \cdots R^{j_m}_{i_m} T_{j_{n+1}, \dots, j_m}^{j_1 \dots j_n} \qty[\vec{f}] \]
\end{definition}

\begin{example}
  The Stress Tensor is a quantity which involves forces acting on a
  three-dimensional object. There are three forces, one acting on a
  plane perpendicular to each plane in the object, and each force
  having three spatial components. The components of the force on the
  plane for which $x$ is the normal are $f_{xx}, f_{xy}, f_{xz}$, so
  the overall force on the body is
  \begin{equation*}
    \begin{pmatrix}
      f_{xx} & f_{xy} & f_{xz} \\ f_{yx} & f_{yy} & f_{yz} \\ f_{zx} & f_{zy} & f_{zz}
    \end{pmatrix}
  \end{equation*}
  and converting to stress, by dividing through by the area of each
  plane,
  \begin{equation*}
    \begin{pmatrix}
      \sigma_{xx} & \sigma_{xy} & \sigma_{xz} \\ \sigma_{yx} & \sigma_{yy} & \sigma_{yz} \\ \sigma_{zx} & \sigma_{zy} & \sigma_{zz}
    \end{pmatrix}
  \end{equation*}
  The elements of the leading diagonal are three orthogonal normal, or
  compressive stresses. The six off-diagonal elements are orthogonal
  sheer stresses, so an alternative notation of the tensor is
  \begin{equation*}
    \begin{pmatrix}
      \sigma_x & \tau_{xy} & \tau_{xz} \\
      \tau_{yx} & \sigma_y & \tau_{yz} \\
      \tau_{xz} & \tau_{yz} & \sigma_z
    \end{pmatrix}
  \end{equation*}
\end{example}

\section{Tensor Rotations}
\label{sec:tensor-rotations}

Rotation tensors are an important quantity in physics, allowing the
transformation between coordinate systems.  Let $(x,y,z)$ be a
coordinate system, and a second coordinate system, $(x^{\prime},
y^{\prime}, z^{\prime})$ is rotated relative to it. The rotation can
be described by a tensor of the form
\begin{equation}
  \label{eq:1}
  \begin{pmatrix}
    x^{\prime} \\ y^{\prime} \\ z^{\prime}
  \end{pmatrix} = 
  \begin{pmatrix}
    \cos(\theta_{xx^{\prime}}) & \cos(\theta_{yx^{\prime}}) & \cos(\theta_{zx^{\prime}})\\
    \cos(\theta_{xy^{\prime}}) & \cos(\theta_{yy^{\prime}}) & \cos(\theta_{zy^{\prime}})\\
    \cos(\theta_{xz^{\prime}}) & \cos(\theta_{yz^{\prime}}) & \cos(\theta_{zz^{\prime}})
  \end{pmatrix}
  \begin{pmatrix}
    x \\ y \\ z
  \end{pmatrix}
\end{equation}

Using the summation convention, and letting the rotation matrix from
equation (\ref{eq:1}) be $a_{ij}$, we can rewrite the rotation
operation as
\[ x_j^{\prime} = a_{ij} x_j \] where $a_{ij} = \hat{e}_i^{\prime}
\cdot \hat{e}_j$.
The rotation is orthonormal, so $A^{-1} = A^{\rm T}$, and so
\[ x_i = a_{ji} x_j^{\prime} \]

\section{Extensions of Rotations to rank-2 Tensors}
\label{sec:extens-rotat-rank}

A tensor can be produced from the outer product of two vectors,

\begin{definition}[Outer Product]
\label{def:outerproduct}
Let $\vec{u}$ and $\vec{v}$ be vectors, then
\[ \vec{u} \otimes \vec{v} = \vec{u} \vec{v}^{\rm T} \] is the inner
product of the two vectors, and is a rank-2 tensor.
\end{definition}

Now, suppose the vectors $\vec{u}$, and $\vec{v}$ are rotated to become
\begin{align*}
  u_i^{\prime} &= a_{ik} u_k \\
  v_j^{\prime} &= a_{jl} v_l
\end{align*}
and we construct a tensor by taking the outer product,
$t^{\prime} = \vec{u}^{\prime} \omult \vec{v}^{\prime}$
then
\[ t^{\prime} = u_i^{\prime} v_j^{\prime} = (a_{ik} u_k)(a_{jl} v_l) =
a_{ik} a_{jl} u_k v_l = a_{ik} a_{jl} t_{kl} \]
and the inverse relation is then
\[ t_{ij} = a_{kl} a_{lj} t^{\prime}_{kl} \] This principle can be
continued for higher and higher order tensors.

\section{The moment of Inertia Tensor}
\label{sec:moment-inert-tens}

Rotational motion of a rigid body depends both on the axis and the
moment of inertia with respect to the axis, with the moment of
inertia, which, for a body composed of masses, $\set{m_i}$ which are
at a distance $r_i$ from the axis can be described
\begin{equation}
  \label{eq:2}
  I = \sum_i m_i r_i^2 = \int \rho r^2 \dd{V}
\end{equation}
In order to have a means of calculating the moment of inertia along
any axis we need a tensor.  In order to find the form of this tensor
we turn to angular momentum.\\
The total angular momentum, $\vec{J}$, of a body is the sum of all the
angular momenta of its constituent parts,
\[ \vec{J} = \sum_i \vec{L}_i = \sum_i \vec{r}_i \times m_i \vec{v}_i
= \sum_i m_i \qty[ \vec{r}_i \times (\omega \times \vec{r}_i)] \] We
assume the body to be rigid, so that $\vec{\omega}$ is constant for all its
constituent particles. Then
{\small
\begin{align*} 
\vec{J} &= \sum_i m_i \qty[ (\vec{r} \cdot \vec{r}) \vec{\omega} -
(\vec{r}_i \cdot \vec{\omega} ) \vec{r}_i] \\
&= \sum_i m_i 
\begin{pmatrix}
  \omega_x (y_i^2 + z_i^2 ) & - \omega_y x_i y_i & - \omega_z x_i z_i \\
  - \omega_x y_i x_i & \omega_y (z_i^2 + x_i^2) & - \omega_z y_i z_i \\
  - \omega_x z_i x_i & - \omega_y z_i y_i & \omega_z (x_i^2 + y_i^2)
\end{pmatrix}\\
&=
\begin{pmatrix}
  \sum_i m_i  (y_i^2 + z_i^2 ) & - \sum_i m_i  x_i y_i & - \sum_i m_i  x_i z_i \\
  - \sum_i m_i  y_i x_i & \sum_i m_i  (z_i^2 + x_i^2) & - \sum_i m_i  y_i z_i \\
  - \sum_i m_i z_i x_i & - \sum_i m_i  z_i y_i & \sum_i m_i  (x_i^2 + y_i^2)
\end{pmatrix}
\begin{pmatrix}
  \omega_x \\ \omega_y \\ \omega_z
\end{pmatrix} \\&= {\cal I} \vec{\omega}
\end{align*}
}
Hence
\begin{definition}[Moment of Inertia Tensor]
  \label{def:momentofinertiatens}
\[
{\cal I} = \begin{pmatrix}
  \sum_i m_i  (y_i^2 + z_i^2 ) & - \sum_i m_i  x_i y_i & - \sum_i m_i  x_i z_i \\
  - \sum_i m_i  y_i x_i & \sum_i m_i  (z_i^2 + x_i^2) & - \sum_i m_i  y_i z_i \\
  - \sum_i m_i z_i x_i & - \sum_i m_i  z_i y_i & \sum_i m_i  (x_i^2 + y_i^2)
\end{pmatrix} 
\]
\end{definition}
The on-diagonal components of the moment of inertia tensor are the
moments of inertia, while the off-diagonal elements are the products
of inertia.

\begin{example}{\em The moment of Inertia Tensor}\\
  A composite body is made up from four masses, $m$, arranged on the
  $x-y$-plane, as shown below.
\begin{center}
\begin{tikzpicture}[]
	\fill [accent-blue] (-1,1) circle (0.2);
	\fill [accent-blue] (0,-1) circle (0.2);
	\fill [accent-blue] (1,1) circle (0.2);
	\fill [accent-blue] (-2,-1) circle (0.2);

	\draw [<->] (-1,1) -- (0,-1) node [midway, right] {$a$};
	\draw [<->] (-2,-1) -- (0,-1) node [midway, right] {$a$};
	\draw [<->] (-1,1) -- (1,1) node [midway, above, xshift=-.3cm] {$a$};
	\draw [<->] (-2,-1) -- (-1,1) node [midway, right] {$a$};
	\draw [<->] (0,-1) -- (1,1) node [midway, right, yshift=-.3cm] {$a$};

	\draw [ultra thick,->] (0,0) -- (3,0) node [below] {$x$};
	\draw [ultra thick,->] (0,0) -- (0,3) node [left] {$y$};
\end{tikzpicture}
\end{center}
We can find the three moments of inertia,
\begin{align*}
  I_{xx} &= \sum m(y^2 + z^2) = ma^2 \qty( 4 \times \frac{3}{16} ) = \frac{3 ma^2}{4}\\
I_{yy} &= \sum m( x^2 + z^2 ) = ma^2 \qty( 2 \times\frac{9}{16} + 2 \times \frac{1}{16} ) = \frac{5 ma^2}{4} \\
I_{zz} &= \sum m(x^2 + y^2 ) = \frac{8 ma^2}{4}
\end{align*}
The products of inertia are straightforward, as $z=0$ causes $I_{xy} =
I_{yz} = 0$, so
\begin{align*}
  I_{xy} &= - \sum m x y = - ma^2 \qty( 2 \times \frac{3}{4} \frac{\sqrt{3}}{4} - 2 \times \frac{1}{4} \frac{\sqrt{3}}{4} ) \\ &= - \frac{\sqrt{3}m a^2}{4}
\end{align*}
So
\[
{\cal I} = \frac{ma^2}{4}
\begin{pmatrix}
  3 & - \sqrt{3} & 0 \\
  - \sqrt{3} & 5 & 0 \\
  0 & 0 & 8
\end{pmatrix}
\]
By diagonalising the matrix we can find the principle moments of inertia. From the characteristic equation of ${\cal I}$,
\begin{align*}
  \chi_{{\cal I}}(\mu) &= \qty( (3-\mu)(5-\mu)-3 ) = 0 \\ &= (2-\mu)(6-\mu) = 0
\end{align*}
so $\mu = \set{2,6,8}$.
Thus
\[ {\cal I} = \frac{ma^2}{4}
\begin{pmatrix}
  2 & 0 & 0 \\ 0 & 6 & 0 \\ 0 & 0 & 8
\end{pmatrix}
\]
So the body's principle moments of inertia are 
\[ I_{\rm prin} = \set{ \half ma^2, \frac{3}{2} ma^2, 2 ma^2} \]

Finally, to find the principle axes we need the corresponding
eigenvectors to the eigenvalues, so

$\mu=2$,
\begin{align*}
  (3-2) x - \sqrt{3} y &= 0 \\
y &= \frac{1}{\sqrt{3}} x
\end{align*}

$\mu=8$,
\begin{align*}
  (3-8)x - \sqrt{3} y &= 0 \\
5x + \sqrt{3} y &= 0 \\
y &= \frac{-5}{\sqrt{3}} x
\end{align*}

$\mu=6$,
\begin{align*}
  (3-6) x - \sqrt{3} y &= 0 \\
y &= \frac{-3}{\sqrt{3}} x = - \sqrt{3} x
\end{align*}

\end{example}

\section{The Parallel Axis Theorem}
\label{sec:parall-axis-theor}

Let ${\cal I}_G$ be the inertia tensor with respect to the centre of
mass, $G$ of a rigid body, and ${\cal I}_O$ be the tensor with respect
to a different point $O$ in the same coordinate frame. Let $\vec{r} =
GO$.
From the definition of the moment of inertia tensor,
\[ {\cal I}_O = {\cal I}_G + M 
\begin{pmatrix}
  r_y^2 + r_z^2 & -r_x r_y & - r_x r_z \\
- r_y r_x & r_z^2 + r_x^2 & - r_y r_z \\
- r_z r_x & - r_z r_y & r_x^2 + r_y^2
\end{pmatrix}
\]
which gives the parallel axis theorem.

% \end{document}
