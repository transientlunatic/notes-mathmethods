\documentclass{mn2e}
\usepackage{danielphysics}
\begin{document}

\section{Lagrangian Mechanics}
\label{sec:lagrangemex}

\begin{definition}[Position]
  The position of a particle in space can be defined by a
  \emph{radius} vector, $\vec{r}$, which describes its location in
  space in reference to a specific, defined point.
\end{definition}
\begin{definition}[Velocity]
  The velocity, $\vec{v}$, of a particle is the time rate of change of its radius vector,
\[ \vec{v} = \dv{\vec{r}}{t} = \dot{\vec{r}} \]
\end{definition}
\begin{definition}[Acceleration]
  The acceleration, $\vec{a}$, of a particle is the time rate of change of its velocity,
\[ \vec{a} = \dv{\vec{v}}{t} = \dv[2]{\vec{r}}{t} = \ddot{\vec{r}}\]
\end{definition}

When presented with a system of $N$ particles in space we need $N$
radius vectors to fully describe the system, and, in 3-space, this
results in needing $3N$ numbers to fully describe the system.

\begin{definition}[Degrees of Freedom]
  The number of degrees of freedom of a system describes the number of
  independent parameters which define the system.
\end{definition}

Thus, we can say that a system of $N$ particles will have $3N$ degrees
of freedom. There is no fundamental requirement to use Cartesian
coordinates in describing a system.

\begin{definition}[Generalised Coordinates]
  For any system with $s$ degrees of freedom, the sequence of quantities
  \[ q_1 + q_2 + \cdots + q_s \] are the generalised coordinates of
  the system, and their derivatives, $\dot{q_i}$, are the generalised
  velocities.
\end{definition}

If a system's coordinates and velocities are fully defined at a given
instant the accelerations are uniquely defined, and it is possible to
predict the future state of the system through equations of motion.

\begin{definition}[Equation of Motion]
  The positions, velocities, and accelerations of a system are related
  through second-order differential equations for the functions $q(t)$.
\end{definition}
The most general equation of motion is the \emph{principle of least
  action}, in which every system is characterised by a definite
function $L(q, \dot{q},t)$.

\begin{definition}[Principle of Least Action]
  Consider a system at two instants, $t_1$ and $t_2$, described by
  coordinates $q^{(1)}$, $q^{(2)}$, and velocities $\dot{q}^{(1)}$,
  $\dot{q}^{(2)}$ respectively.
\end{definition}

\end{document}
