\message{ !name(complex.tex)}
\message{ !name(complex.tex) !offset(-2) }
% \documentclass{dwnotes}
% \usepackage{danielphysics}
%\begin{document}
\section{Complex Analysis}
\label{sec:complex}

Complex analysis is the study of functions of complex variables.

\begin{definition}[Complex Function]
  A complex function is a function which maps a complex number to
  another complex number.
\end{definition}
Any complex function, like a complex number, can be split into real
and imaginary components, so, a function $w: \mathbb{C} \to
\mathbb{C}$ can be composed of two functions, $u: \mathbb{R} \to
\mathbb{R}$, and $v: \mathbb{R} \to \mathbb{R}$, so
\[ w(z) = u(x,y) + i v(x,y) \]

\begin{definition}[Multi-valued Function]
  A multi-valued function is a left-total relation, such that each
  member in the domain of the relation corresponds to at least one
  member in the co-domain.
\end{definition}
The name ``function'' for these objects is a misnomer, and they are in
fact relations, since functions must always map to only a single
value.
\begin{example}
  The relation $w^2 = z$ is multi-valued, because, for example, both
  $w=\pm 1$ map to the same value; in fact every output value greater
  than 0 is mapped to by two numbers.
\end{example}
Dealing with multi-valued functions requires a new framework compared
to normal functions. One approach is to consider the \emph{branching}
of a function.
\begin{definition}[Branch Point]
  The branch points of a multi-valued function are the points in the
  complex plane where the complex argument can be mapped from a single
  point in the domain to multiple points in the range of the relation.
\end{definition}
\begin{definition}[Winding Number]
  The winding number for a closed curve on a plane around a point is
  an integer which represents the number of closed circuits the curve
  makes around the point.
\end{definition}
\begin{example}
  Let $w(\theta) = z(\theta)^{\half}$, with $z \in \mathbb{C}$, such that $z = e^{i
    \theta}$. Taking the value of $w$ at two points, $0$, and $2 \pi$,
  we find
  \[ w(0) = e^0 = 1 \] and \[ w(2\pi) = e^{2 \pi i} ( \neq 1) \]
each corresponds to the same point in the domain, but to different points in the range.
\[ w(4 \pi) = 1 \] so the function returns to its original value after
two circuits about the complex plane, so this point has a winding number of two.
\end{example}
The existence of branch points splits multi-valued functions into
``sheets'' in which the function is single valued. It can be useful,
in order to work with these functions, to consider a multi-valued
function as the ``glueing together'' of such single value
functions. These gluings happen along ``branch cuts.''
\begin{definition}[Branch Cut]
  A branch cut is a curve in the complex plane, such that a
  multi-valued function could be represented as a single valued
  function as a branch minus the curve.
\end{definition}
\begin{example}
  Returning to the example of $w = z^{\half}$, a branch cut would
  exist as the line where $\theta = 2 \pi$. All the values on the
  branch below this are single valued.
\end{example}
There is an alternative way of considering multi-valued functions, and
one which is easier to visualise. The different ``sheets'' of a
multivalued function can be considered as being superimposed ontop of
each other. If we were able to seperate the two superimposed sheets,
we would need to leave them somehow connected along the branch
cuts. The resulting surface is a Riemann Surface.
\begin{example}
  Again, in the case $w = z^{\half}$, the Riemann surface consists of
  two surfaces. The lower surface is joined to the upper surface at
  $\theta=2 \pi$, and the upper surface is joined back to the lower
  surface along $\theta=4 \pi$ and $\theta=0$.
\end{example}

\subsection{Limits and Continuity}
\label{sec:limits}

\begin{definition}[Limit (Complex)]
  Let $f$ be a function of a complex variable, $z$, which is defined
  and single-valued in a neighbourhood about a point $z=z_0$ in the
  complex plane, with the possible exception of $z = z_0$, thus a
  deleted neighbourhood of $z_0$. The number $l$ is the limit of
  $f(z)$ as $z$ approaches $z_0$, if, for any small positive number
  $\epsilon$ there exists a $\delta$, such that \[ | f(z) - l | <
  \epsilon \qquad \text{when} \qquad 0 < |z - z_0 | <
  \delta \] Then \[ \lim_{z \to z_0} f(z) = l \]
\end{definition}
\begin{definition}[Continuity]
  Let $f$ be a function of a complex variable, $z$, which is
  single-valued in the neighbourhood of $z=z_0$ and at the point
  $z=z_0$. The function $f$ is continuous at $z=z_0$ if $\lim_{z \to
    z_0} f(z) = f(z_0)$. This implies three conditions,
  \begin{itemize}
  \item $\lim_{z \to z_0} f(z) = l$ must exist
  \item $f(z_0)$ must exist
  \item $l = f(z_0)$
  \end{itemize}
\end{definition}
\begin{definition}[Continuity over a region]
  A function is said to be continuous over a region if it is
  continuous at all points in that region. 
\end{definition}
\begin{definition}[Uniform Continuity]
  Let $f$ be a function of a complex variable which is continuous in a
  region. At each point, $z_0$ of the region, and for any $\epsilon
  >0$ there is a $\delta >0$ such that $|f(z) - f(z_0)| < \epsilon$
  when $|z - z_0| < \delta$. If there is a $\delta$ depending on
  $\epsilon$ but not on $z_0$ $f(z)$ is uniformly continuous in the
  region.
\end{definition}

\subsection{Complex Differentiation}
\label{sec:complexdiff}

\begin{definition}[Differentiation]
  Let $f$ be a function of a complex variable, $z$, which is
  single-valued over some region ${\cal R}$ of the $z$ plane, then the
  derivative of $f$ is
  \[ \dv{f}{z} = \lim_{\Delta z \to 0} \frac{f(z + \Delta z) -
    f(z)}{\Delta z} \] provided that the limit exists independent of
  the way in which $\Delta z \to 0$. In such a case $f$ is
  differentiable at $z$.
\end{definition}

\begin{definition}[Holomorphic Function]
  Let $f$ be a function of a complex variable $z$ which has a
  derivative, $f^{\prime}$, which exists at every point in a region
  ${\cal R}$, the $f$ is a holomorphic function in ${\cal R}$.
\end{definition}
\begin{example}
  The function $f(z) = \bar{z}$ is not holomorphic, because
  approaching $0$ from different directions, the limit in the
  definition of the complex derivative is different.
\end{example}
The condition that a function is holomorphic is given by the
Cauchy-Riemann equations.
\begin{theorem}[Cauchy-Riemann Equations]
  Consider a function, $f$ of a complex variable, $z$, such that
  \[ w = f(z) = u(x,y) + i v(x,y) \] for $x,y \in \mathbb{R}$. If the
  function satisfies the relations
  \[ \pdv{u}{x} = \pdv{u}{y} \] and \[ \pdv{u}{y} = - \pdv{v}{x} \]
  which are the \emph{Cauchy-Riemann equations}, the function is
  holomorphic.
\end{theorem}
An important property of holomorphic functions relates to the
existence of higher derivatives.
\begin{theorem}[Higher Derivatives]
  Suppose $f$ is a holomorphic function over a region ${\cal R}$. The
  derivatives $f^{\prime}, f^{\prime \prime}, \dots$ are all holomorphic
  over ${\cal R}$.
\end{theorem}

It is possible for functions to have points where they fail to be analytic.
\begin{definition}[Singularity]
  A singularity is a point where a function on the complex plane
  ceases to be analytic.\\
  \textbf{Isolated singularities} exist at a point $z=z_0$ where it is
  possible to find a circle $|z-z_0|=\delta$ for $\delta>0$ encloses
  no singular point other than $z_0$. If no such $\delta$ exists the
  singularity is non-isolated. \\
  \textbf{Poles} are isolated singularities where there exists an
  integer $n$ such that $\lim_{z \to z_0} (z -z_0)^n f(z) = A \neq 0$;
  such a pole is of order $n$. If $n=1$ the pole is a simple pole.\\
  \textbf{Branch Points} are non-isolated singular points since a
  multi-valued function is not continuous, and thus is not holomorphic
  in the deleted neighbourhood of the branch point.\\
  \textbf{Removable Singularities} are isolated singular points, $z_0$
  where $\lim_{z \to z_0} f(z)$ exists. Defining $z_0 = \lim_{z \to
    z_0} f(z)$ the function becomes continuous and holomorphic at
  $z_0$.  \\ \textbf{Essential Singularities} are singularities which
  are neither poles nor removable singularities.
\end{definition}

\subsection{Complex Integration}
\label{sec:complexint}

\begin{definition}[Complex Line Integral]
  Let $f(z)$ be a function of a complex variable which is continuous
  at all points of a curve, $C$, which has a finite length. The line
  integral along the curve can be found by dividing the curve into $n$
  sections by means of the points $z_1, z_2, \dots, z_{n-1}$ which are
  chosen arbitrarily. Let $a = z_0$ and $b=z_n$ be the beginning and
  end points respectively of the curve. On each arc joining points
  $z_{k-1}$ to $z_k$, for $k = 1, 2, \dots, n$, choose a point
  $\xi_k$. A sum can then be formed, letting $\Delta z_k = z_k - z_{k-1}$,
  \[S_n = \Delta z_1 f(\xi_1) + \Delta z_2 f(\xi_2) + \cdots + \Delta
  z_n f(\xi_n) \] As $|\Delta z_k| \to 0$, we have the integral,
  \[ \int_a^b f(z) \dd{z} = \int_C f(z) \]
\end{definition}

\begin{definition}[Simply and Mulitply Connected Regions]
  A region ${\cal R}$ can be described as simply connected if any
  closed curve which lies in ${\cal R}$ can be shrunk to a point
  without leaving ${\cal R}$. Otherwise the region is multiply
  connected.
\end{definition}

\begin{definition}[Jordan Curve]
  A curve $C$ is a Jordan curve if itis continuous, closed, and does
  not intersect itself.
\end{definition}

\begin{theorem}[Jordan Curve Theorem]
  A Jordan curve on the complex plane divides the plane into two
  regions, having the curve as a common boundary.  The region which is
  bounded (where points do not lie on the boundary) is the interior of
  the curve, while the other region os the exterior of the curve.
\end{theorem}

%%%%% By rights there's plenty more which should go in here, but it
%%%%% wasn't covered in the course, so I'll hold off on putting it in
%%%%% for now.

\subsection{Residues}
\label{sec:residues}


%\end{document}

\message{ !name(complex.tex) !offset(-229) }
