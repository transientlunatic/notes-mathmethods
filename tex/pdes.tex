\section{Einstein Summation Convention}
\label{sec:einsteinsummation}

Consider a three-dimensional vector space, $\mathsf{V}$, over the
field of real numbers, $\mathbb{R}$.  Any point in the space can be
described by an ordered set of three numbers, $(x_1, x_2, x_3)$, known
as coordinates, such that
\[ \vec{A} = (x_1, x_2, x_3) \cdot
\begin{pmatrix} \vec{e_1} \\ \vec {e_2} \\ \vec{e_3}
\end{pmatrix}
\]
where $\vec{A}$ is any vector in $\mathsf{V}$, and $\vec{e_1}$, $\vec
{e_2}$, and $\vec{e_3}$ constitute a basis for $\mathsf{V}$.

This can be expressed in a more compact form by adopting the {\em
  Einstein summation convention}. In this system the summation sign,
$\sum$, is omitted, and replaced with repeated indices, viz.
\[ \vec{A} = A_i \vec{e_i} = \sum_{i=1}^3 A_i\vec{e_i} \] here $i$ is
a repeated index, and so the summation over it is implicit.  This
allows the definition of a number of vector calculus operations in a
more compact way.

\subsection{Operations}
\label{sec:operations}


\subsubsection{Dot Product}
\label{sec:dotproduct}

\begin{equation}
  \label{eq:dotprod}
  \vec{a} \cdot \vec{b} = a_\theta b_\theta
\end{equation}

\subsubsection{Total Differential}
\label{sec:totaldiff}

\begin{equation}
  \label{eq:totaldifferential}
  \dif{f} = \frac{\partial f}{\partial x_\theta} \dif{x_\theta}
\end{equation}

\subsubsection{Matrix Multiplication}
\label{sec:matrixmult}

\begin{equation}
  \label{eq:matmult}
  a_{ij} = b_{i \theta} c_{\theta j}
\end{equation}

\subsubsection{Cross Product}
\label{sec:crossprod}

\begin{equation}
  \label{eq:crossprod1}
  [ \vec a \times \vec b]_i = \epsilon_{i \theta \phi} a_{\theta} b_{\phi}
\end{equation}
\begin{equation}
  \label{eq:crossprod2}
  \vec{a} \times \vec{b} = \epsilon_{i \theta \phi} a_{\theta} b_{\phi} \vec{e_{i}}
\end{equation}
Here $\epsilon_{i \theta \phi}$ is the Levi-Civita tensor, and takes
values

\begin{equation}
  \label{eq:levitcivita}
  \epsilon_{i \theta \phi} = \left\{ 
    \begin{array}{rl}
      0 & \text{if any indices are repeated.} \\
      1 & \text{if indices are a cyclic permutation of } (1,2,3). \\
      -1 & \text{if indices are a cyclic permutation of } (1,3,2). 
    \end{array}\right.
\end{equation}
There is a simple relationship between this tensor and the Kronecker
delta,
\begin{equation}
  \label{eq:deltalevi}
  \epsilon_{\theta j k } \epsilon_{\theta l m} = \delta_{jl} \delta_{k m} - \delta_{j m} \delta_{l k}
\end{equation}

\subsection{Del Operators}
\label{sec:deloperators}
The gradient is now expressed:
\begin{equation}
  \label{eq:grad}
  \nabla = \vec e_i \frac{\partial }{\partial x_i}
\end{equation}
The divergence:
\begin{equation}
  \label{eq:divergence}
  \nabla \cdot \vec{A}(\vec{r}) = \frac{\partial A_i(\vec{r})}{\partial x_i}
\end{equation}
and the curl:
\begin{equation}
  \label{eq:curl}
  \nabla \times \vec{A}(\vec{r}) = \epsilon_{ijk} \frac{\partial A_j(\vec{r})}{\partial x_i} \vec{e_k}
\end{equation}
\begin{example}
  Consider
  \[ \left[ \vec{a} \times ( \vec{b} \times \vec{c} ) \right]_m \]
  using the summation convention,
  \begin{align*} \left[ \vec{a} \times ( \vec{b} \times \vec{c} )
    \right]_m
    &= \epsilon_{m \alpha \beta} a_{\alpha} \epsilon_{\beta \gamma \delta} b_{\gamma} c_{\delta} \\
    &= \epsilon_{\beta m \alpha} \epsilon_{\beta \gamma \delta} a_{\alpha} b_{\gamma} c_{\delta} \\
    &= (\delta_{m \delta}\delta_{\alpha \delta} - \delta_{m \delta} \delta_{a \gamma}) a_{\alpha} b_{\gamma} c_{\delta}\\
    &=  (a_{\alpha}c_{\alpha}) b_m - (a_{\alpha}b_{\alpha}) c_m \\
    &= \left[ (\vec a \cdot \vec c) \vec b - (\vec a \cdot \vec b)
      \vec c \right]_{m}
  \end{align*}
\end{example}
Note, \[ \delta_{i\theta} a_{\theta} = a_i \]
\begin{example}
  \begin{align*}
    \nabla \times ( \nabla \phi ) &= \epsilon_{m\alpha \beta} \frac{\partial }{\partial x_a} \frac{\partial \phi}{\partial x_{\beta}}\\
    &= 0
  \end{align*}
  This arises because the Levi-Civita tensor is anti-symmetric, but
  the two partial derivatives are symmetric.
\end{example}
\begin{example}
  \begin{align*}
    \vec A \cdot \vec B \times \vec C &= A_{\alpha}\epsilon_{\alpha \beta \gamma} B_{\beta} C_{\gamma} \\
    &= C_{\gamma} \epsilon_{\gamma \alpha \beta} A_{\alpha} B_{\beta} \\
    &= C_{\gamma} [\vec A \times \vec B]_{\gamma} \\
    &= C \cdot [ \vec A \times \vec B]
  \end{align*}
\end{example}

\section{Curvilinear coordinates}
\label{sec:curvi}

A curvilinear coordinate system is a coordinate system over a space in
which the coordinate lines may be curved, and are related to a
Cartesian coordinate system by a bijection.  In physics, laws are
independent of reference frame and of coordinate system, which allows
the use of the most appropriate coordinate system for a specific
situation---for example, if a problem has spherical symmetry it is
likely to be easiest to solve in spherical polar coordinates.

\subsection{Spherical Polar Coordinates}
\label{sec:sphpol}

\begin{minipage}[!]{\linewidth}
  \noindent\begin{minipage}[!]{0.45\linewidth}
    \definecolor{accent-pink}{HTML}{C3325F}
\definecolor{accent-red}{HTML}{9d261d}
\definecolor{accent-purple}{HTML}{7a43b6}
\definecolor{muted-blue}{HTML}{195073}
\definecolor{muted-green}{HTML}{7F8C1F}
\definecolor{muted-orange}{HTML}{EE913F}
\begin{tikzpicture}
  \draw [->, thick, accent-red] (0,0) -- (0,2.5);
  \draw [->, thick, accent-red] (0,0) -- (-1, -1.5);
  \draw [->, thick, accent-red] (0,0) -- (3,0);
 
  \draw [->>, ultra thick, accent-purple] (0,0) -- (2,2);
  \draw [dashed, muted-orange] (2,2) -- (2, -1);
  \draw [<->] (0,0) -- (2,-1) node [below,] {$r$};
 
  \draw [<->] (0,0.7) arc (45:0:0.7) node [yshift=8pt,above]{$\theta$};
  \draw [<->] (.5, -.3) arc (300:255:1) node [yshift=-2pt, below] {$\phi$};
\end{tikzpicture}
  \end{minipage}
  \hfill
  \noindent\begin{minipage}[!]{0.45\linewidth}
    They are related to Cartesian coordinates by the transformations:
    \begin{align}
      x &= r \sin \theta \cos \phi \label{eq:sph1}\\
      y &= r \sin \theta \sin \phi \label{eq:sph2}\\
      z &= r \cos \theta \label{eq:sph3}
    \end{align}
    \hfill \vfill
  \end{minipage}
\end{minipage}

\subsection{Cylindrical Polar Coordinates}
\label{sec:cylpol}
\begin{minipage}[!]{\linewidth}
  \noindent\begin{minipage}[!]{0.45\linewidth}
    \definecolor{accent-pink}{HTML}{C3325F}
\definecolor{accent-red}{HTML}{9d261d}
\definecolor{accent-purple}{HTML}{7a43b6}
\definecolor{muted-blue}{HTML}{195073}
\definecolor{muted-green}{HTML}{7F8C1F}
\definecolor{muted-orange}{HTML}{EE913F}
\begin{tikzpicture}
  \draw [->, thick, accent-red] (0,0) -- (0,2.5);
  \draw [->, thick, accent-red] (0,0) -- (-1, -1.5);
  \draw [->, thick, accent-red] (0,0) -- (3,0);
 
  \draw [->>, ultra thick, accent-purple] (0,2) -- (2,1.5);
  \draw [dashed, muted-orange] (2,1.5) -- (2, -1);
  \draw [<->] (0,0) -- (2,-1) node [below,] {$r$};
  \draw [<->] (-.1,0) -- (-.1,2) node [left] {$z$};
  \draw [<->] (.5, -.3) arc (300:255:1) node [yshift=-2pt, below] {$\phi$};
\end{tikzpicture}
  \end{minipage}
  \hfill
  \noindent\begin{minipage}[!]{0.45\linewidth}
    They are related to Cartesian coordinates by the transformations:
    \begin{align}
      x &= r \cos \theta \label{eq:cyl1}\\
      y &= r \sin \theta \label{eq:cyl2}\\
      z &= z \label{eq:cyl3}
    \end{align}
    \hfill \vfill
  \end{minipage}
\end{minipage}

\subsection{Unit vectors and scale factors}
\label{sec:unitsscale}

In any coordinate system there is a concept of distance, which is
independent of the choice of coordinate system.  In Cartesian
coordinates this is
\[ \dif{s}^2 = \dif{x}^2 + \dif{y}^2 + \dif{z}^2 = \dif{r}\cdot
\dif{r} \] where $\dif{r}$ is the inifinitessimal line element.

Consider a general coordinate system described by $q_i$, related to
Cartesian coordinates via
\[ x_i = f_i (q_1, q_2, q_3) \] As $q_i$ is changed the position
vector $\vec r$ will move:
\[ \frac{\partial \vec r}{\partial q_i} = h_{q_i} \vec{e}_{q_i} \] The
magnitude of $\frac{\partial \vec r}{\partial q_i}$ is the scale
factor, $h_{q_i}$, and the basis vector is the unit vector in the
direction of $\frac{\partial \vec r}{\partial q_i}$.  We can then
define the infinitessimal line element,
\begin{equation}
  \label{eq:infinitessimalcurve}
  \dif{s}^2 = \sum_{i, j} g_{ij} \dif{q_i} \dif{q_j}
\end{equation}
Where $g_{ij}$ is the metric for the geometry we are considering.  The
volume element is then
\begin{equation}
  \label{eq:volumecurvi}
  \dif{V} = \dif{s}_{1} + \dif{s}_2 + \dif{s}_3 = h_{q_1}h_{q_2}h_{q_3} \dif{q_1} \dif{q_2} \dif{q_3}
\end{equation}
\begin{example}[Spherical Polar Coordinates]
  From the relations in equations (\ref{eq:sph1}) to (\ref{eq:sph3}),
  and considering that
  \[ \vec{r} = x \vec{e_x} + y \vec{e_y} + z \vec{e_{z}} \] Then,
  \begin{align*}
    \frac{\partial \vec r}{\partial r}
    &= \frac{\partial x}{\partial r} \vec{e_x} + \frac{\partial y}{\partial r} \vec{e_y} + \frac{\partial z}{\partial r} \vec{e_z} \\
    &= \sin \theta \cos \phi \vec{e_{x}} + \sin\theta \sin \phi \vec{e_y} + \cos \theta \vec{e_z} = h_r\vec{e_r} \\
    \frac{\partial \vec r}{\partial \theta}
    &= \frac{\partial x}{\partial \theta} \vec{e_x} + \frac{\partial y}{\partial \theta} \vec{e_y} + \frac{\partial z}{\partial \theta} \vec{e_z} \\
    &= r \cos \theta \cos \phi \vec{e_x} + r \cos \theta \sin \phi \vec{e_y} - r \sin \theta \vec{e_z} = h_{\theta}\vec{e_{\theta}} \\
    \frac{\partial \vec r}{\partial \phi}
    &= \frac{\partial x}{\partial \phi} \vec{e_x} + \frac{\partial y}{\partial \phi} \vec{e_y} + \frac{\partial z}{\partial \phi} \vec{e_z} \\
    &= - \sin \theta \sin \phi \vec{e_{x}} + r \sin\theta \cos \phi \vec{e_y} = h_{\phi}\vec{e_{\phi}} \\
  \end{align*}
  Then the scale factors are
  \begin{align*}
    h_r &= \left| \frac{\partial \vec r}{\partial r} \right| = 1\\
    h_{\theta} &= \left| \frac{\partial \vec r}{\partial \theta} \right| = r\\
    h_{\phi} &= \left|  \frac{\partial \vec r}{\partial \phi} \right| = r \sin \theta\\
  \end{align*}
  Also, the volume element,
  \[ \dif{V} = r^2 \sin \theta \dif{r} \dif{\theta} \dif{\phi} \] and
  the square of the infinitessimal line element,
  \[ \dif{s}^2 = \dif{r}^2 + r^2 \dif{\theta}^2 + r^2 \sin^2 \theta
  \dif{\phi^2} \]
\end{example}

\subsection{Del Operators}
\label{sec:delcurvi}

\subsubsection{Gradient}
\label{sec:gradcurvi}

The gradient in the direction of $\vec{e_{q_1}}$ is
\begin{equation} \label{eq:dirgradcurvi} \nabla_{q_i} =
  \frac{\partial}{\partial s_i} = \frac{1}{h_{q_i}}
  \frac{\partial}{\partial q_i}
\end{equation}
thus, the total gradient operator is
\begin{equation}
  \label{eq:gradincurvi}
  \nabla = \sum_i \vec{e_{q_i}} \frac{\partial}{\partial s_i} = \sum_i \vec{e_{q_i}} \frac{1}{h_{q_i}} \frac{\partial}{\partial q_i}
\end{equation}

\subsubsection{Divergence}
\label{sec:divcurvi}

We can define the divergence from the gradient, since
\[ \nabla \cdot \vec{A} = \nabla \cdot \left( \sum_j A_j \vec{e_{q_j}}
\right) \] clearly we need $\nabla \cdot \vec{e_{q_j}}$ to continue;
from equation (\ref{eq:dirgradcurvi}),
\[ \nabla \cdot q_j = \sum_i \frac{1}{h_{q_i}} \frac{\partial
  q_j}{\partial q_i} = \sum_i \vec{e_{q_i}} \frac{1}{h_{q_i}}
\delta_{ij} = \vec{e_{q_j}} \frac{1}{h_{q_j}} \] Thus,
\[\vec{e_{q_j}} = h_{q_j} \nabla q_j \]
The basis vectors form a right-handed set, so,
\[e_{q_1} \times e_{q_2} = e_{q_3} \] and
\[\nabla q_1 \times \nabla q_2 =
\frac{\vec{e_{q_3}}}{h_{q_1}h_{q_2}}\] Considering that $\curl(\grad)
= 0$,
\begin{align*} \nabla \cdot (\nabla_{q_1} \times \nabla_{q_2}) &=
  \nabla q_2 \cdot (\nabla \times \nabla q_1) - \nabla q_1 \cdot (
  \nabla \times \nabla q_2) \\ &= 0
\end{align*}
So
\[ \nabla \cdot \left( \frac{\vec{e_{q_3}}}{h_{q_1} h_{q_2}} \right) =
0 \] and the same argument applies for cyclic permutations,
\[ \nabla \cdot \left( \frac{\vec{e_{q_3}}}{h_{q_1} h_{q_2}} \right) =
\nabla \cdot \left( \frac{\vec{e_{q_1}}}{h_{q_2} h_{q_3}} \right) =
\nabla \cdot \left( \frac{\vec{e_{q_2}}}{h_{q_3} h_{q_1}} \right) =
0 \] Then returning to
\begin{align*}
  \nabla \cdot \vec{A} &= \nabla \cdot \left( \sum_j A_j \vec{e_{q_j}} \right) &\\
  &= \nabla \cdot \left\{ [h_{q_1}h_{q_2}A_1] \left[
      \frac{\vec{e_{q_1}}}{h_{q_2} h_{q_3}} \right] \right.
  &+& [h_{q_3}h_{q_1}A_2] \left[ \frac{\vec{e_{q_2}}}{h_{q_3} h_{q_1}} \right] \\
  &&+&\left.[h_{q_1} h_{q_2}A_3]  \left[ \frac{\vec{e_{q_3}}}{h_{q_1} h_{q_2}} \right]  \right\} \\
  &= \left[ \frac{\vec{e_{q_1}}}{h_{q_2} h_{q_3}} \right] \cdot \nabla
  [h_{q_1}h_{q_2}A_1]
  &+&\left[ \frac{\vec{e_{q_2}}}{h_{q_3} h_{q_1}} \right] \cdot \nabla [h_{q_3}h_{q_1}A_2]\\
  &&+&\left[ \frac{\vec{e_{q_3}}}{h_{q_1} h_{q_2}} \right]\cdot \nabla
  [h_{q_1}h_{q_2}A_3]
\end{align*}
Since
\[\vec{e_{q_i}} \cdot \nabla = \frac{1}{h_{q_i}}
\frac{\partial}{\partial q_i}\],
\begin{equation}
  \label{eq:divincurvi}
  \begin{split}
    \nabla \cdot \vec{A} = \frac{1}{h_{q_1}h_{q_2}h_{q_3}} \left( \frac{\partial}{\partial q_1} (h_{q_2} h_{q_3} A_1) \right. \\
    + \left. \frac{\partial}{\partial q_2} (h_{q_3} h_{q_1} A_2) +
      \frac{\partial}{\partial q_3} (h_{q_2} h_{q_1} A_3) \right)
  \end{split}
\end{equation}
\subsubsection{Curl}
\label{sec:curlcurvi}
The curl can be derived from earlier results,
\begin{align*}
  \nabla \times \vec{A} &=&& \nabla \times \left( \sum_j A_j \vec{e_{q_j}} \right) \\
  &=&& \nabla \times \left(
    \begin{matrix}
      [h_{q_1}A_1] \qty[\frac{\vec{e_{q_1}}}{h_{q_1}} ] &+
      [h_{q_2}A_2] \qty[ \frac{\vec{e_{q_2}}}{h_{q_2}} ] \\ &+
      [h_{q_3}A_3] \qty[ \frac{\vec{e_{q_3}}}{h_{q_3}} ]
    \end{matrix}
  \right) \\
  &= && - \left[ \frac{\vec{e_{q_1}}}{h_{q_1}} \right] \times \nabla
  (h_{q_1}A_1)
  - \left[ \frac{\vec{e_{q_2}}}{h_{q_2}} \right] \times \nabla (h_{q_2}A_2) \\
  &&& - \left[ \frac{\vec{e_{q_3}}}{h_{q_3}} \right] \times \nabla
  (h_{q_3}A_3)
\end{align*}
so,
\begin{equation}
  \label{eq:curlincurvi}
  \nabla \times \vec{A} = 
  \frac{1}{h_{q_1}h_{q_2}h_{q_3}}
  \begin{vmatrix}
    h_{q_1} \vec{e_1}              & h_{q_2} \vec{e_2}              & h_{q_3} \vec{e_{3}}             \\
    \frac{\partial}{\partial q_1} & \frac{\partial}{\partial q_2} & \frac{\partial}{\partial q_3} \\
    h_{q_1} A_1 & h_{q_2} A_2 & h_{q_3} A_3
  \end{vmatrix}
\end{equation}

\subsubsection{Laplacian}
\label{sec:laplaciancurvi}

\begin{equation}
  \label{eq:laplacianincurvi}
  \begin{split}
    \nabla^2 = \frac{1}{h_{q_1}h_{q_2}h_{q_3}} 
    \bigg( \pdv{q_1} \qty[ \frac{h_{q_2}h_{q_3}}{h_{q_1}} \pdv{q_1} ] 
    +  \pdv{q_2} \qty[ \frac{h_{q_3}h_{q_1}}{h_{q_2}} \pdv{q_2} ] \\
    +  \pdv{q_3} \qty[ \frac{h_{q_2}h_{q_1}}{h_{q_1}} \pdv{q_3} ] \bigg) 
  \end{split}
\end{equation}
\section{Partial Differential Equations in Curvlinear Coordinate
  Systems}
\label{sec:curvipde}

\subsection{Common Partial Differential Equations}
\label{sec:commonpdes}

There are a number of common PDEs which it is useful to know.

\subsubsection{Laplace's Equation} \label{sec:laplacepde}

\begin{equation}
  \label{eq:laplace}
  \nabla^2 \phi(\vec{r}) = 0
\end{equation}
This equation is used in electromagnetism, gravitation, hydrodynamics,
and heat flow in situations where no sources or sinks exist.

\subsubsection{Poisson's Equation} \label{sec:poissonpde}

\begin{equation}
  \label{eq:poisson}
  \nabla^2 \phi(\vec{r}) = f(\vec{r})
\end{equation}
This is used in the same situations as Laplace's equation,
(\ref{eq:laplace}), only when there {\em are} sources or sinks,
described by the scalar field $f$.
\begin{example}
  One of Maxwell's equations is
  \[ \nabla \cdot \vec{E} = \frac{\rho(\vec{r})}{\epsilon_{0}} \] with
  electric field $\vec E$, charge density $\rho(\vec r)$, and the
  permittivity of free space, $\epsilon_0$. Since $\vec{E} = - \nabla
  \phi$, we have
  \[ \nabla^2 \phi(\vec{r}) = - \frac{\rho(\vec{r})}{\epsilon_0} \]
\end{example}

\subsubsection{Diffusion Equation} \label{sec:diffusionpde}

\begin{equation}
  \label{eq:diffusion}
  \nabla^2 \phi(\vec{r}, t) = \frac{1}{\alpha} \frac{\partial \phi(\vec{r}, t}{\partial t}
\end{equation}
The diffusion equation describes the time and space evolution of
fields where there is no source; $\phi$ would describe the
distribution of temperature in a conductive heat flow situation, for
example.
\begin{example}
  Consider heat flowing into a metal, with the temperature a scalar
  field, represented by a function of position, $\vec{r}$, and time,
  $t$, so $T(\vec{r}, t)$. Then the heat in a small volume, $V$, is
  \[ Q = \int_V \rho c_{\rm p} T(\vec{r}, t) \difp{3}{\vec{r}} \] The
  rate at which heat transfers from one volume to another depends on
  the temperature gradient, the area of the contact, and the metal's
  thermal conductivity. For a boundary of area $A$,
  \[ \frac{\dif{Q}}{\dif{t}} = \int_A k \dif{\vec{\sigma}} \cdot
  \nabla T(\vec{r}, t) \] with $\dif{\vec{\sigma}}$ the normal vector
  to the area, $\dif{A}$. Applying the divergence theorem,
  \begin{align*}
    \frac{\dif{Q}}{\dif{t}} &= \int_V \nabla \cdot [k \nabla
    T(\vec{r}, t)] \difp{3}{\vec{r}} \\
    &= \int_V k \nabla^2 T(\vec{r}, t) \difp{3}{\vec{r}}
  \end{align*}
  Then equating the expressions for $\frac{\dif{Q}}{\dif{t}}$, and
  assuming $\rho$ and $c_{\rm p}$ are constant,
  \begin{align*}
    \frac{\dif{Q}}{\dif{t}} &= \int_V \nabla \cdot [k \nabla
    T(\vec{r}, t) ] \difp{3}{\vec{r}} \\ &= \int_{V} k \nabla^2
    T(\vec{r}, t) \difp{3}{\vec{r}} \\ \nabla^2 T(\vec{r}, t) &=
    \frac{\rho c_{\rm p}}{k} \frac{\partial T(\vec{r},t)}{\partial t}
  \end{align*}
\end{example}

\subsubsection{Wave Equation}
\label{sec:wavepde}

\begin{equation}
  \label{eq:wave}
  \nabla^2 \phi(\vec{r}, t) = \frac{1}{v^2} \frac{\partial^2 \phi(\vec{r}, t)}{\partial t^2}
\end{equation}
The wave equation describes the progression of vibrations through
media. It occurs frequently in physics, and an operator is defined for
it, the d'Alembertian operator,
\begin{equation}\Box^2 \equiv \frac{1}{v^2}
  \frac{\partial^2 \phi(\vec{r}, t)}{\partial t^2}
\end{equation}

\subsubsection{Helmholtz Equation}
\label{sec:helmholtzpde}

\begin{equation}
  \label{eq:helmholtz}
  \nabla^2 \phi + k^2 \phi = 0
\end{equation}

This appears where the time dependence of the diffusion equation is
removed by the separation of variables.

\subsubsection{Schrodinger Equation}
\label{sec:schrodingerpde}

Time-independent:
\begin{equation}
  \label{eq:tindyschrodinger}
  - \frac{\hbar^2}{2m} \nabla^2 \psi + V \psi = E \psi
\end{equation}
Time-dependent:
\begin{equation}
  \label{eq:tdschrodinger}
  - \frac{\hbar^2}{2m} \nabla^2 \psi + V \psi = i \hbar \frac{\partial \psi}{\partial t}
\end{equation}

\subsection{Method of Separation of Variables}
\label{sec:sepvar}

Consider the diffusion equation, (\ref{eq:diffusion}), and look for
solutions with the form \[ \phi(\vec{r}, t) = \Phi(\vec{r}) T(t) \]
Solutions of this form exist for these equations because they are
linear and have no cross-terms.  So we now have
\begin{equation*}
  \frac{1}{\Phi (\vec{r}) } \nabla^2 \Phi(\vec{r}) = \frac{1}{\alpha} \frac{1}{T(t)} \dv{T(t)}{t}  = -k^2
\end{equation*}
since both sides must be equal, they are also both equal to a
constant.  So now we have
\begin{align*}
  \frac{\dif{T}}{\dif{t}} &= -\alpha k^2 T \\
  \nabla^2 \Phi(\vec{r}) &= -k^2 \Phi(\vec{r})
\end{align*}
For examples, see MM1 notes.

\subsection{Separation of Variables in spherical and cylindrical
  coordinate systems}
\label{sec:sepvarcylsph}

We can decompose Laplace's equation, equation (\ref{eq:laplace}), into
three equations.  In spherical coordinates Laplace's equation becomes
\begin{equation}
  \begin{split}
    \frac{1}{r^2 \sin \theta} 
    \bigg( \pdv{r}      \left[r^2 \sin \theta \pdv{\psi(r, \theta, \phi)}{r} \right] \\ 
         + \pdv{\theta} \left[\sin \theta \pdv{\psi(r, \theta, \phi)}{\theta}\right] \\ 
         + \pdv{\phi}   \left[\frac{1}{\sin \theta} \pdv{\psi(r,\theta, \phi)}{\psi } \right] \bigg)\\ =0
    \end{split}
  \end{equation}
  Then, rewriting $\psi$ as a product of three functions,
  \[ \psi(r, \theta, \phi) = R(r)\Theta(\theta)\Phi(\phi) \] and
  dividing by this,
  \begin{equation*}
    \begin{split}
      \frac{1}{R \Theta \Phi} 
      \bigg( \Theta \Phi \frac{\partial}{\partial r} \left[r^2 \sin \theta \frac{\partial \psi(r, \theta, \phi)}{\partial r} \right] \\
           + R \Phi \frac{\partial}{\partial \theta} \left[ \sin \theta \frac{\partial \psi(r, \theta, \phi)}{\partial \theta}\right] \\ 
           + R \Theta \frac{\partial}{\partial \phi} \left[ \frac{1}{\sin \theta} \frac{\partial \psi(r, \theta, \phi)}{\partial \psi }\right]
      \bigg) \\=0
      \end{split}
    \end{equation*}

    \begin{equation*}
      \begin{split}
        \frac{1}{R} \frac{\partial}{\partial r} \left[r^2 \sin \theta
          \frac{\partial \psi(r, \theta, \phi)}{\partial r} \right] \\
        + \frac{1}{\Theta} \frac{\partial}{\partial \theta} \left[
          \sin \theta \frac{\partial \psi(r, \theta, \phi)}{\partial
            \theta}\right] \\ + \frac{1}{\Phi}
        \frac{\partial}{\partial \phi} \left[ \frac{1}{\sin \theta}
          \frac{\partial \psi(r, \theta, \phi)}{\partial \psi }\right]
        \left.\vphantom{\frac{1}{2}}\right) \\=0
      \end{split}
    \end{equation*}
    Then
    \begin{equation*}
      \sin^2\theta \frac{1}{R} \frac{\partial}{\partial r} \left[ r^2 \frac{\partial R}{\partial r} \right]
      +\sin \theta \frac{1}{\Theta} \frac{\partial}{\partial \theta} \left[ \sin \theta \frac{\partial \Theta}{\partial \theta}\right] 
      = - \frac{1}{\Phi} \frac{\difp{2}{\Phi}}{\dif \Phi^2}
    \end{equation*}
    Each side must be constant, so
    \begin{align*}
      \frac{1}{\Phi} \dv[2]{\Phi}{\phi} &= -m^2 \\
      \sin^2(\theta) \frac{1}{R} \dv{r} \qty[r^2 \dv{R}{r}] +
      \sin(\theta) \frac{1}{\Theta} \dv{\theta} \qty[\sin(\theta)
      \dv{\Theta}{\theta}] &= m^2
    \end{align*}
    and these can themselves be made equal to a constant,
    \begin{align*}
      \frac{1}{R} \dv{r} \qty[r^2 \dv{R}{r}] &= -
      \frac{1}{\sin(\theta)} \frac{1}{\Theta} \dv{\theta}
      \qty[\sin(\theta) \dv{\Theta}{\theta}] +
      \frac{m^2}{\sin^2(\theta)} = k
    \end{align*}
    {\em The $R(r)$ equation:}\\
    \begin{align*}
      \dv{r} \qty[r^2 \dv{R(r)}{r}] - kR(r) &= 0 \\
      r^2 \dv[2]{R(r)}{r} + 2r \dv{R(r)}{r} - kR(r) &=0
    \end{align*}
    The coefficients are polynomials of $r$, so we try a solution of
    the form $R(r) = r^n$,
    \begin{align*}
      r^2n(n-1) r^{n-2} + 2r nr^{n-1} - kr^n &= 0 \\
      n(n-1)r^n + 2nr^n - kr^n &= 0 \\
      n(n+1) - k &= 0
    \end{align*}
    and now let $k = l(l+1)$, so $n(n+1) = l(l+1)$, thus $n=l$ or $n =
    -l-1$ making the general solution
    \begin{equation}
      \label{eq:generalsolr}
      R(r) = A r^l + B r^{-l-1}
    \end{equation}
    {\em The equation for $\Phi(\phi)$}\\
    \[ \dv[2]{\Phi(\phi)}{\phi}= -m^2 \Phi(\phi) \] The solution must
    have the form $\Phi(\phi) = e^{\alpha \phi}$, so
    \[ \alpha^2 \Phi(\phi) = -m^2 \Phi(\phi) \] so $\alpha = \pm
    e^{\alpha \phi}$ . Thus, the general solution is
    \begin{equation}
      \label{eq:generalsolphi}
      \Phi(\phi) = A^{\prime} \sin(m \phi) + B^{\prime} \cos(m \phi)
    \end{equation}
    for constants $A^{\prime}$, and B$^{\prime}$, and $m \in
    \mathbb{N}$, since $\Phi(\phi + 2 \pi) = \Phi(\phi)$.

    {\em The equation for $\Theta(\theta)$}\\
    \[ \sin(\theta) \dv{\theta} \qty[ \sin(\theta)
    \dv{\Theta(\theta)}{\theta}] - m^2 \Theta(\theta) + l(l+1)
    \sin^2(\theta) \Theta(\theta) = 0 \] which has the form of the
    Associate Legendre Differential equation, equation
    (\ref{eq:assoclegendrede}), and the solution is therefore an
    Associate Legendre polynomial, with a general solution of the form
    \[ \Theta(\theta) = A P_l^m \qty( \cos(\theta) ) \] Thus, the
    general solution of
    \[ \nabla^2 \psi(r, \theta, \phi) = 0 \] is
    \begin{equation}
      \label{eq:solutionpdesphere}
      \psi(r, \theta, \phi) = \sum_{l=0}^{\infty} \sum_{m=0}^l (A_{lm}r^l + B_{lm}r^{-l-1}) P_l^m \qty( \cos \theta) e^{\pm i m \phi}
    \end{equation}
    $A$, and $B$ are constants determined by the boundary conditions
    of the problem. The functions
    \begin{equation}
      \label{eq:sphericalharm}
      Y_l^m(\theta, \phi) = N e^{im\phi} P_l^m (\cos(\theta))
    \end{equation}
    with $l \ge 0$, and $|m| \le l$, are {\em Spherical Harmonics}.

\begin{example}
  The surface of a metal sphere of radius $r_0$ is held at an
  electrostatic potential of $V_0 \cos(\theta)$, with $\theta$ the
  polar angle. What is the electrostatic potential inside and outside
  the sphere, assuming no other sources of charge?
  %\tdplotsetmaincoords{70}{135}
  % \begin{tikzpicture}[scale=3,opacity=0.0,tdplot_main_coords,fill
  %   opacity=0.7]
  %   \tdplotsphericalsurfaceplot[parametricfill]{72}{36}%
  %   {cos(\tdplottheta)}{transparent!0}{\tdplotphi}%
  %\end{tikzpicture}
  There are no sources other than the sphere, so outside the sphere
  the potential obeys Laplace's equation. The problem has spherical
  symmetry.\\
  {\em Inside the sphere}\\
  $R(r) \propto r^{-l-1}$ would give an infinity at $r=0$ , so
  $B_{lm}=0$.  Thus,
  \[
  \psi(r, \theta, \phi) = \sum_{l=0}^{\infty} \sum_{m=0}^l A_{lm}r^l
  P_l^m \qty( \cos \theta) e^{\pm i m \phi}
  \]
  and we have the boundary condition that
  \[ \psi(r_0, \theta, \phi) = \phi_0 \cos(\theta) \] which has no
  $\phi$ dependence, implying $m=0$.  Then,
  \[ \psi(r, \theta, \phi) = \sum_{l=0}^{\infty} A_{l}r^l P_l \qty(
  \cos \theta) = \phi_0 \cos(\theta) \] so only $l=1$ can satisfy this
  equation, and thus the potential is
  \begin{equation*}
    \psi(r, \theta, \phi) = \psi_0 \frac{r}{r_0} \cos(\theta)
  \end{equation*}
  {\em Outside the sphere}\\
  The same arrguments apply outside the sphere as did inside, so,
  \[ R(r) \propto r^l \] would give an infinity as $r \to
  \infty$. Thus $A_{lm}=0$. The same arguments for angular depenence
  also apply, so,
  \begin{equation*}
    \psi(r, \theta, \phi) = \psi_0 \qty( \frac{r_0}{r})^2 \cos(\theta)
  \end{equation*}
\end{example}

%%% Local Variables: 
%%% mode: latex
%%% TeX-master: "notes"
%%% End: 
