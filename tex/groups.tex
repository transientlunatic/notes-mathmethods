% \documentclass[]{dwnotes}
% \usepackage{danielphysics}
% \usepackage{danielgroups}
% \title{Group Theory}
% \author{Daniel Williams}

% \begin{document}\maketitle

% \begin{abstract}
%   Group theory is the study of abstract structures known as
%   groups. Many physical systems are modelled by symmetry groups,
%   e.g. crystals and the Hydrogen atom.
%\end{abstract}

\section{Group Theory}
\label{sec:groups}


\subsection{Basic Concepts}
\label{sec:intro}

\begin{definition}[Group]
  A group is an abstract mathematical entity which is composed of a
  set, $G$, with an associated binary operation, $*$. In order to be a
  group, the pair $(G, *)$ must satisfy the follwing axioms,
  \begin{enumerate}
  \item For all $a,b \in G$, $a*b \in G$
    (closure). \label{itm:groupclosure} \item For all $a,b,c, \in G$, $(a*b)*c = a*(b*c)$ (associativity). \label{itm:groupassoc}
\item There exists $e \in G$, such that, for every $a \in G$, $a*e = e*a = a$ holds (identity). \label{itm:groupidentity}
  \item For each $a \in G$ there exists a $b \in G$ such that $a*b = b*a = e$ (inverse). \label{itm:groupinverse}
  \end{enumerate}
\end{definition}
\begin{definition}[Group Order]
  The order of a group is the number of elements it contains. If the
  group has finite order the group is described as a finite group; if
  it has an infinite number of elements it is an infinite group.
\end{definition}
\begin{example}
  The hours on a clock represent a group, with a set $H =
  \{1,2,3,4,5,6,7,8,9,10,11,12\}$, and an operation, addition $\mod
  12$. The order of this group is 12, as it contains 12 elements.
\end{example}
\begin{definition}[Homomorphism]
  Given two groups, $(G,*)$ and $(H, \cdot)$, a group homomorphism
  grom $(G,*)$ to $(H, \cdot)$ is a function $h|G \to H$, such that,
  for all $u$ and $v$ in $G$ it holds
  \[ h(u * v) = h(u)\cdot h(v) \]
\end{definition}
\begin{definition}[Isomophism]
  A group isomorphism is a function between two groups which sets up a
  bijection between the elements of the groups in a way which respects
  the given group operations.\\
  Given two groups, $(G,*)$ and $(H, \cdot)$, a group isomorphism from
  $(G,*)$ to $(H,\cdot)$ is a bijective group homomorphism from $G$ to
  $H$, that is, an isomorphism is a bijective function $f : G \to H$,
  such that, for all $u,v \in G$, it holds \[ f(u*v) = f(u) \cdot
  f(v) \] If such a function exists we can write
  \[(G,*) \cong (H,\cdot) \]
\end{definition}
\begin{example}
  The group of real numbers under addition, and the group of real
  numbers under multiplication are isomorphic under the bijection \[
  f(x) = e^x \]
\end{example}
\begin{definition}[Abelian Group]
  A group, $(G,*)$ is called Abelian, if, in addition to the axioms
  for a group, it also satisfies a commutative property;
  \begin{enumerate}
  \setcounter{enumi}{4}
\item For all $a,b \in G$, $a*b = b*a$.
  \end{enumerate}
\end{definition}
\begin{definition}[Subgroup]
  A set $H$ which is a subset of $G$, where $(G,*)$ is a group, is
  called a subgroup iff $(H,*)$ is a group, and \[ H \le G \] A
  subgroup, $H$, is a \emph{trivial} subgroup if the group, $G$, has
  only the identity element. Otherwise, if $H \neq G$ then $H$ is a
  \emph{proper} subgroup.
\end{definition}
\begin{definition}[Generating Set]
  The generating set of a group is a subset such that, any element of
  the group can be expressed as the combination of finitely many
  elements of the subset and their inverses.
\end{definition}
\begin{definition}[Cayley Diagram]
  Suppose that $G$ is a group, and $S$ is a generating set for
  $G$. They Cayley diagram, $\Gamma = \Gamma(G,S)$ is a coloured
  directed graph with the construction:
  \begin{itemize}
  \item each element $g\in G$ is assigned a vertex. The vertex set
    $V(\Gamma)$ of $\Gamma$ is thus identified with $G$.
  \item  each generator is assigned a colour, $c_s$
  \item For any $g\in G, s \in S$, the vertices corresponding to the elements $g$ and $gs$ are joined by a directed edge of colour $c_s$, and thus the edge set, $E(\Gamma)$ is composed of the pairs of form $(g,gs)$, with $s\in S$ providing the colour.
  \end{itemize}
For simplicity, the identity element is omitted, leaving a normal graph without loops.
\end{definition}

\subsection{Finite Groups}
\label{sec:finiteexamples}

\subsubsection{Cyclic Groups}
\label{sec:cyclicgroups}

\begin{definition}[Cyclic Group]
  A group, $G$, is called cyclic if there exists an element $g$ from
  $G$ such that every element in $G$ can be obtained by repeatedly
  applying the group operation to $g$ or its inverse.
\end{definition}
The cyclic groups are an important simple group, and describe the
rotational symmetries of regular polyhedra.
\begin{figure}
  \centering
  \begin{tikzpicture}
    \CayleyCyclic{4}
   \end{tikzpicture}
  \caption{A Cayley diagram of Z(4).}
  \label{fig:cayleycyclic}
\end{figure}

\subsubsection{Symmetric Groups}
\label{sec:symmetricgroups}

\begin{definition}[Symmetric Group]
  A symmetric group on a finite set, $X$ is a group whose elements are all bijective functions from $X$ to $X$, and with the operation of function composition.
\end{definition}
\begin{figure}
  \centering
  \begin{tikzpicture}
    \CayleySymmetric{4}
  \end{tikzpicture}
  \caption{A Cayley Diagram of S4.}
  \label{fig:cayleysymmetric}
\end{figure}

\subsubsection{Dihedral Groups}
\label{sec:dihedralgroups}

\begin{definition}[Dihedral Groups]
  A regular polygon with $n$ sides has $2n$ symmetries; $n$
  rotational, and $n$ reflective symmetries. The rotations and
  reflections which preserve these symmetries compose the elements of the dihedral group of order $n$, ${\rm D}_n$.
\end{definition}
\begin{figure}
  \centering
  \begin{tikzpicture}
    \CayleyDihedral{4}
   \end{tikzpicture}
  \caption{A Cayley diagram of ${\rm D}_4$.}
  \label{fig:cayleydihedral}
\end{figure}

\subsection{Continuous Groups}
\label{sec:contgroups}

The continuous, or \emph{Lie} groups, are groups which are composed of
an infinite set equipped with a binary operation. Lie groups are also
differentiable manifolds, with the property that the group operation
is compatible with the smooth structure of the manifold. They are
named after Sophus Lie, who laid the foundations for their study. Of
particular interest to physics are the \emph{classical groups}, all of which are closely related to symmetry in Euclidean spaces. There are seven classical groups; 
\begin{itemize}
\item general linear---GL($n$)
\item special linear---SL($n$)
\item orthogonal---O($n$)
\item special orthogonal---SO($n$)
\item unitary---U($n$)
\item special unitary---SU($n$)
\item symplectic---Sp($n$)
\end{itemize}

\subsubsection{GL($n$)---The General Linear Group}
\label{sec:genlinear}

${\rm GL}(n)$, \emph{The General linear group of degree $n$}, are the
set of $n\times n$ invertible matrices, equipped with the operation of
matrix multiplication.

\subsubsection{SU($n$)---The Special Unitary Group}
\label{sec:su3}

${\rm SU}(n)$, the \emph{Special unitary group of degree $n$}, are
composed of the set of $n\times n$ unitary (i.e. $UU* = U*U = I$)
matrices with determinant
1, equipped with the operation of matrix multiplication.\\
These are important in physics, as they do not affect the norm of the
vector quantity on which they operate.

\paragraph{SU(2)}
The generators of SU(2) are the Pauli matrices,
\begin{equation*}
  \begin{matrix}
    \sigma_1 = \begin{pmatrix} 0 & 1 \\ 1 & 0 \end{pmatrix} &
    \sigma_2 = \begin{pmatrix}  0 & -i \\ i & 0 \end{pmatrix} &
    \sigma_3 = \begin{pmatrix}  1 & 0 \\ 0 & -1 \end{pmatrix}
  \end{matrix}
\end{equation*}
These matrices act on the \emph{spinors}, 
\begin{equation*}
  \begin{matrix}
    u = \begin{pmatrix} 1 \\ 0 \end{pmatrix} &
    d = \begin{pmatrix} 0 \\ 1 \end{pmatrix}
  \end{matrix}
\end{equation*}
which represent the spin up and spin down states. Then, the quantum
mechanical spin operator can be related to these via 
\begin{equation}
  \label{eq:spinoperator}
  \hat{S}_i = \frac{\hbar}{2} \sigma_i
\end{equation}
\paragraph{SU(3)}
The generators of SU(3) are
the \emph{Gell-Mann} matrices, $\lambda_{1,\dots,8}$.
\begin{equation*}
  \begin{matrix}
    \lambda_1 = \begin{pmatrix} 0 & 1 & 0 \\ 1 & 0 & 0 \\ 0 & 0 & 0  \end{pmatrix} &
    \lambda_2 = \begin{pmatrix} 0 & -i &0 \\ i & 0 & 0 \\ 0 & 0 & 0  \end{pmatrix} \\
    \lambda_3 = \begin{pmatrix} 1 & 0 & 0 \\ 0 & -1 & 0 \\ 0 & 0 & 0 \end{pmatrix} &
    \lambda_4 = \begin{pmatrix} 0 & 0 & 1 \\ 0 & 0 & 0 \\ 1 & 0 & 0  \end{pmatrix} \\
    \lambda_5 = \begin{pmatrix} 0 & 0 &-i \\ 0 & 0 & 0 \\ i & 0 & 0  \end{pmatrix} &
    \lambda_6 = \begin{pmatrix} 0 & 0 & 0 \\ 0 & 0 & 1 \\ 0 & 1 & 0  \end{pmatrix} \\
    \lambda_7 = \begin{pmatrix} 0 & 0 & 0 \\0 & 0 & -i \\ 0 & i & 0  \end{pmatrix} &
    \lambda_8 = \frac{1}{\sqrt{3}} \begin{pmatrix} 1 & 0 & 0 \\ 0 & 1 & 0 \\ 0 & 0 & -2 \end{pmatrix}
  \end{matrix}
\end{equation*}
These obey the relations
\begin{subequations}
\begin{equation}
  \label{eq:gellmanncommutator}
  [T_a , T_b ] = i f_{abc} T_c
\end{equation}
\begin{equation}
  \label{eq:gellmannanticomm}
  \{T_a, T_b \} = \frac{1}{3} \delta_{ab} + d_{abc} T_c
\end{equation}
\end{subequations}
where $T_a = \frac{\lambda_a}{2}$, and $f_{abc}, d_{abc}$ are the
structure constant tensors.\\
The Pauli matrices act on the spinors
\begin{equation*}
  \begin{matrix}
    u = \begin{pmatrix}  1 \\ 0  \\ 0  \end{pmatrix} &
    d = \begin{pmatrix}  0 \\ 1 \\ 0   \end{pmatrix} &
    s = \begin{pmatrix}  0 \\ 0 \\ 1   \end{pmatrix}
  \end{matrix}
\end{equation*}
representing the up, down, and strange states, and the isospin raising
and lowering operators can be defined,
\begin{equation}
  \label{eq:isospinraise}
  \hat{I}_{\pm} = \half (\lambda_1 \pm i \lambda_2)
\end{equation}
and the isospin projection operator, $I_3$,
\begin{equation}
  \label{eq:isospinprojoperator}
  \hat{I}_3 = \half \lambda_3 
\end{equation}
Similarly, the operators
\begin{subequations}
  \begin{equation}
    \label{eq:ushift}
    \hat{U}_{\pm} = \half (\lambda_6 \pm i \lambda_7)
  \end{equation}
  \begin{equation}
    \label{eq:vshift}
    \hat{V}_{\pm} = \half ( \lambda_4 \mp i \lambda_5)
  \end{equation}
\end{subequations}
Both of these, combined with their respective projection operators,
$\hat{U}_3$ and $\hat{V}_3$,
\begin{subequations}
  \begin{equation}
    \label{eq:uproj}
    \hat{U}_3 = - \frac{1}{4} \lambda_3 + \frac{\sqrt{3}}{4} \lambda_8
  \end{equation}
  \begin{equation}
    \label{eq:vproj}
    \hat{V}_3 = - \frac{1}{4} \lambda_3 - \frac{\sqrt{3}}{4} \lambda_8
  \end{equation}
\end{subequations}
 define two different SU(2) subgroup
representations of SU(3).
%\end{document}


